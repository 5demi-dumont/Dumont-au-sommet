\renewcommand{\thechapter}{\Roman{chapter}}	
\renewcommand{\chaptermark}[1]{%
	\markboth{\thechapter{} — #1}{}%
}
\makeatletter
\renewcommand{\numberline}[1]{#1\hspace{0.7em}} % espace après le numéro
\makeatother
\pagestyle{fancy}

\fancyhf{}
\lhead{\leftmark}
\chead{}
\rhead{\copyright Arthur Esteve, Lorenzo Marconot}
\lfoot{MP Dumont d'Urville}
\cfoot{}
\rfoot{\thepage/\pageref*{LastPage}}

\fancypagestyle{plain}{%
	\fancyhf{}
	\lhead{\leftmark}
	\chead{}
	\rhead{}
	\lfoot{}
	\cfoot{}
	\rfoot{\thepage/\pageref*{LastPage}}
}
\setlength{\parindent}{0pt}





\newcommand{\N}{\mathbb{N}}
\newcommand{\Z}{\mathbb{Z}}
\newcommand{\D}{\mathbb{D}}
\newcommand{\Q}{\mathbb{Q}}
\newcommand{\R}{\mathbb{R}}
\newcommand{\C}{\mathbb{C}}
\newcommand{\K}{\mathbb{K}}
\newcommand{\U}{\mathbb{U}}
\newcommand{\p}{\mathbb{P}}
\newcommand{\E}{\mathbb{E}}
\newcommand{\V}{\mathbb{V}}
\newcommand{\B}{\mathcal{B}}
\newcommand{\A}{\mathcal{A}}
\newcommand{\F}{\mathcal{F}}
\newcommand{\Or}{\mathcal{O}}
\newcommand{\M}{\mathcal{M}}
\newcommand{\norme}[1]{\left\|#1\right\|}
\newcommand{\normep}[2]{{\left\|#2\right\|}_{#1}}
\newcommand{\normetriple}[1]{\mid\mid\mid #1 \mid\mid\mid}
\newcommand{\ens}[1]{\lbrace#1\rbrace}
\newcommand{\crblanc}[2]{\llbracket #1;#2 \rrbracket}
\newcommand{\proscal}[2]{\langle #1,#2\rangle}
\newcommand{\vect}{\text{Vect}}
\newcommand{\Tr}{\operatorname{Tr}}
\newcommand{\rg}{\operatorname{rg}}
\newcommand{\ch}{\operatorname{ch}}
\newcommand{\sh}{\operatorname{sh}}
\newcommand{\cotan}{\operatorname{cotan}}
\newcommand{\pgcd}{\operatorname{pgcd}}
\newcommand{\ppcm}{\operatorname{ppcm}}
\newcommand{\Card}{\operatorname{Card}}
\newcommand{\diag}{\operatorname{diag}}
\newcommand{\Id}{\operatorname{Id}}
\newcommand{\Com}{\operatorname{Com}}
\newcommand{\Fr}{\operatorname{Fr}}
\newcommand{\Ker}{\text{Ker}}
\newcommand{\Ima}{\text{Im}}
\newcommand{\Vect}{\operatorname{Vect}}
\newcommand{\T}{\text}
\newcommand{\Mat}{\operatorname{Mat}}
\newcommand{\fonction}[5]{
	#1:\left \{
	\begin{array}{rcl}
		#2 & \longrightarrow & #3 \\
		#4 & \longmapsto     & #5 \\
	\end{array}
	\right.}
\newcommand{\bigO}[1]{\ensuremath{\mathop{}\mathopen{}\mathcal{O}\mathopen{}\left(#1\right)}}
\newcommand{\smallo}[1]{\ensuremath{\mathop{}\mathopen{}{\scriptstyle\mathcal{O}}\mathopen{}\left(#1\right)}}
\newcommand{\nfty}{n\to +\infty}
\newcommand{\unfty}[1]{\underset{n\to +\infty}{#1}}
\newcommand{\ukfty}[1]{\underset{k\to +\infty}{#1}}
\newcommand{\upfty}[1]{\underset{p\to +\infty}{#1}}
\newcommand{\uxfty}[1]{\underset{x\to +\infty}{#1}}
\newcommand{\utfty}[1]{\underset{t\to +\infty}{#1}}
\newcommand{\uxzero}[1]{\underset{x\to 0}{#1}}
\newcommand{\utzero}[1]{\underset{t\to 0}{#1}}
\newcommand{\independent}{\protect\mathpalette{\protect\independenT}{\perp}}
\def\independenT#1#2{\mathrel{\rlap{$#1#2$}\mkern2mu{#1#2}}}
\newcommand{\dpartial}[3]{\displaystyle\frac{\partial^#2 #1}{\partial#3^#2}}
\newcommand{\dpartials}[3]{\displaystyle\frac{\partial^2 #1}{\partial #2\partial #3}}
\usetikzlibrary{shapes.geometric}

\newcommand{\etoile}[2][fill=Yellow,draw=Orange]{
	\begin{tikzpicture}[baseline=-0.35em,#1]
		\foreach \X in {1,...,#2} {
			\pgfmathsetmacro{\xfill}{min(1,max(1+#2-\X,0))}
			\path (\X*1.1em,0)
			node[star,draw,star point height=0.25em,minimum size=0.85em,inner sep=0pt,
			path picture={\fill (path picture bounding box.south west)
				rectangle  ([xshift=\xfill*1em]path picture bounding box.north west);}]{};
		}
\end{tikzpicture}}

\newcommand{\ccinp}[2][fill=BurntOrange!50,draw=BurntOrange]{
	\begin{tikzpicture}[baseline=-0.35em,#1]
		\foreach \X in {1,...,#2} {
			\pgfmathsetmacro{\xfill}{min(1,max(1+#2-\X,0))}
			\path (\X*1.1em,0)
			node[star,draw,star point height=0.25em,minimum size=0.85em,inner sep=0pt,
			path picture={\fill (path picture bounding box.south west)
				rectangle  ([xshift=\xfill*1em]path picture bounding box.north west);}]{};
		}
	\end{tikzpicture}
}


\newcommand{\centraleponts}[2][fill=Red!50,draw=Red]{
	\begin{tikzpicture}[baseline=-0.35em,#1]
		\foreach \X in {1,...,#2} {
			\pgfmathsetmacro{\xfill}{min(1,max(1+#2-\X,0))}
			\path (\X*1.1em,0)
			node[star,draw,star point height=0.25em,minimum size=0.85em,inner sep=0pt,
			path picture={\fill (path picture bounding box.south west)
				rectangle  ([xshift=\xfill*1em]path picture bounding box.north west);}]{};
		}
	\end{tikzpicture}
}
\newcommand{\xens}[2][fill=MidnightBlue!50,draw=MidnightBlue]{
	\begin{tikzpicture}[baseline=-0.35em,#1]
		\foreach \X in {1,...,#2} {
			\pgfmathsetmacro{\xfill}{min(1,max(1+#2-\X,0))}
			\path (\X*1.1em,0)
			node[star,draw,star point height=0.25em,minimum size=0.85em,inner sep=0pt,
			path picture={\fill (path picture bounding box.south west)
				rectangle  ([xshift=\xfill*1em]path picture bounding box.north west);}]{};
		}
	\end{tikzpicture}
}

\newcommand{\telecom}[2][fill=Green!50,draw=Green]{
	\begin{tikzpicture}[baseline=-0.35em,#1]
		\foreach \X in {1,...,#2} {
			\pgfmathsetmacro{\xfill}{min(1,max(1+#2-\X,0))}
			\path (\X*1.1em,0)
			node[star,draw,star point height=0.25em,minimum size=0.85em,inner sep=0pt,
			path picture={\fill (path picture bounding box.south west)
				rectangle  ([xshift=\xfill*1em]path picture bounding box.north west);}]{};
		}
	\end{tikzpicture}
}